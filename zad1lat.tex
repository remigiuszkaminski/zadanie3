\documentclass[12pt,a4paper]{article}
\usepackage[MeX]{polski} 
\usepackage[utf8]{inputenc}  
\usepackage{longtable}
\usepackage{graphicx}
\usepackage{hyperref}
\graphicspath{ {./images/} }
\title{Title praca}
\author{Remigiusz Kamiński}
\date{30.01.2022}
\begin{document}
\maketitle
\newpage
\begin{abstract}
 Streszczenie tego dokumentu.
	W bagnie żył olbrzym Shrek, którego cenna samotność została nagle zakłócona inwazją dokuczliwych postaci z bajek. Ślepe myszki buszują w zapasach olbrzyma, zły wilk sypia w jego łóżku, a trzy świnki buszują po jego samotni. Wszystkie te postaci zostały wypędzone ze swego królestwa przez złego Lorda Farquaada.
	
\end{abstract}
\newpage
\tableofcontents

\newpage
\section{Dlaczego kochamy Shreka}
\begin{center}
\begin{tabular}{ |c|c|c| } 
 \hline
 Shrek & Fiona & osioł \\ 
 grzybek & kot & Ciastek \\ 
 farquad & kopciuch & ja \\ 
 \hline
\end{tabular}
\end{center}
	\subsection{Za co mozna go nie kochac?}
		\includegraphics{shrek2.jpg}
		\\\textit {Otoz trzeba go kochac za uczciwość, uczynność, ugodowość, uprzejmość, uwaga (uważność), wesołość, wiarygodność, wielkoduszność, wierność, wrażliwość, wspaniałomyślność, wyrozumiałość, wytrwałość, zdecydowanie, zdyscyplinowanie, zrównoważenie, życzliwość}
		\label{ogr}
		\\\textit {Ogr o imieniu Shrek chce za wszelką cenę odzyskać spokój na terenie swojej posiadłości na bagnach, gdzie w wyniku represji okrutnego Lorda Farquaada zesłane zostały różne bajkowe postacie, w tym Pinokio, Wilk czy siedmiu krasnoludków. Shrek decyduje się na wyprawę do siedziby Farquaada, miasta Duloc, by odebrać prawa do swoich ziem, a w konsekwencji odzyskać utracony spokój. W wyprawie towarzyszy mu niezdarny Osioł. W wyniku negocjacji zawarty zostaje układ: w zamian za otrzymanie dokumentu ogr zobowiązuje się uwolnić ze smoczej wieży piękną królewnę Fionę, którą Lord wybrał na swoją przyszłą małżonkę.\cite{5}
Po dotarciu do smoczej wieży Shrek odbija Fionę i wychodzi cało z potyczki ze smokiem. Podczas drogi powrotnej nawiązuje się uczucie pomiędzy Shrekiem a Fioną. Żadne z nich nie zdobyło się jednak na miłosne wyznanie. W międzyczasie ujawniona zostaje tajemnica skrywana przez królewnę – Fiona po zachodzie słońca w konsekwencji rzuconego na nią uroku przemienia się w ogra. Fiona, mimo uczucia, jakim darzy Shreka, decyduje się na zaślubiny z Lordem Farquaadem, gdyż wierzy, że pocałunek prawdziwej miłości z Lordem przerwie urok rzucony na nią.
Na skutek mobilizacji ze strony Osła Shrek rusza do Duloc, by wyznać uczucie Fionie. W ostatniej chwili przerywa ceremonię zaślubin, ofiarując swą miłość Fionie, która zmienia się na stałe w ogra, natomiast Lord Farquaad zostaje zjedzony przez smoka, który uwolnił się z wieży. Shrek i Fiona biorą ślub.Ogr o imieniu Shrek chce za wszelką cenę odzyskać spokój na terenie swojej posiadłości na bagnach, gdzie w wyniku represji okrutnego Lorda Farquaada zesłane zostały różne bajkowe postacie, w tym Pinokio, Wilk czy siedmiu krasnoludków. Shrek decyduje się na wyprawę do siedziby Farquaada, miasta Duloc, by odebrać prawa do swoich ziem, a w konsekwencji odzyskać utracony spokój. W wyprawie towarzyszy mu niezdarny Osioł. W wyniku negocjacji zawarty zostaje układ: w zamian za otrzymanie dokumentu ogr zobowiązuje się uwolnić ze smoczej wieży piękną królewnę Fionę, którą Lord wybrał na swoją przyszłą małżonkę.
Po dotarciu do smoczej wieży Shrek odbija Fionę i wychodzi cało z potyczki ze smokiem. Podczas drogi powrotnej nawiązuje się uczucie pomiędzy Shrekiem a Fioną. Żadne z nich nie zdobyło się jednak na miłosne wyznanie. W międzyczasie ujawniona zostaje tajemnica skrywana przez królewnę – Fiona po zachodzie słońca w konsekwencji rzuconego na nią uroku przemienia się w ogra. Fiona, mimo uczucia, jakim darzy Shreka, decyduje się na zaślubiny z Lordem Farquaadem, gdyż wierzy, że pocałunek prawdziwej miłości z Lordem przerwie urok rzucony na nią.
Na skutek mobilizacji ze strony Osła Shrek rusza do Duloc, by wyznać uczucie Fionie. W ostatniej chwili przerywa ceremonię zaślubin, ofiarując swą miłość Fionie, która zmienia się na stałe w ogra, natomiast Lord Farquaad zostaje zjedzony przez smoka, który uwolnił się z wieży. Shrek i Fiona biorą ślub. znać uczucie Fionie. W ostatniej chwili przerywa ceremonię zaślubin, ofiarując swą miłość Fionie, która zmienia się na stałe w ogra, natomiast Lord Farquaad zostaje zjedzony przez smoka, który uwolnił się z wieży. Shrek i Fiona biorą ślub.Ogr o imieniu Shrek chce za wszelką cenę odzyskać spokój na terenie swojej posiadłości na bagnach, gdzie w wyniku represji okrutnego Lorda Farquaada zesłane zostały różne bajkowe postacie, w tym Pinokio, Wilk czy siedmiu krasnoludków. Shrek decyduje się na wyprawę do siedziby Farquaada, miasta Duloc, by odebrać prawa do swoich ziem, a w konsekwencji odzyskać utracony spokój. W wyprawie towarzyszy mu niezdarny Osioł. W wyniku negocjacji zawarty zostaje układ: w zamian za otrzymanie dokumentu ogr zobowiązuje się uwolnić ze smoczej wieży piękną królewnę Fionę, którą Lord wybrał na swoją przyszłą małżonkę.\cite{1}
Po dotarciu do smoczej wieży Shrek odbija Fionę i wychodzi cało z potyczki ze smokiem. Podczas drogi powrotnej nawiązuje się uczucie pomiędzy Shrekiem a Fioną. Żadne z nich nie zdobyło się jednak na miłosne wyznanie. W międzyczasie ujawniona zostaje tajemnica skrywana przez królewnę – Fiona po zachodzie słońca w konsekwencji rzuconego na nią uroku przemienia się w ogra. Fiona, mimo uczucia, jakim darzy Shreka, decyduje się na zaślubiny z Lordem Farquaadem, gdyż wierzy, że pocałunek prawdziwej miłości z Lordem przerwie urok rzucony na nią.
Na skutek mobilizacji ze strony Osła Shrek rusza do Duloc, by wyznać uczucie Fionie. W ostatniej chwili przerywa ceremonię zaślubin, ofiarując swą miłość Fionie, która zmienia się na stałe w ogra, natomiast Lord Farquaad zostaje zjedzony przez smoka, który uwolnił się z wieży. Shrek i Fiona biorą śluznać uczucie Fionie. W ostatniej chwili przerywa ceremonię zaślubin, ofiarując swą miłość Fionie, która zmienia się na stałe w ogra, natomiast Lord Farquaad zostaje zjedzony przez smoka, który uwolnił się z wieży. Shrek i Fiona biorą ślub.Ogr o imieniu Shrek chce za wszelką cenę odzyskać spokój na terenie swojej posiadłości na bagnach, gdzie w wyniku represji okrutnego Lorda Farquaada zesłane zostały różne bajkowe postacie, w tym Pinokio, Wilk czy siedmiu krasnoludków. Shrek decyduje się na wyprawę do siedziby Farquaada, miasta Duloc, by odebrać prawa do swoich ziem, a w konsekwencji odzyskać utracony spokój. W wyprawie towarzyszy mu niezdarny Osioł. W wyniku negocjacji zawarty zostaje układ: w zamian za otrzymanie dokumentu ogr zobowiązuje się uwolnić ze smoczej wieży piękną królewnę Fionę, którą Lord wybrał na swoją przyszłą małżonkę.
Po dotarciu do smoczej wieży Shrek odbija Fionę i wychodzi cało z potyczki ze smokiem. Podczas drogi powrotnej nawiązuje się uczucie pomiędzy Shrekiem a Fioną. Żadne z nich nie zdobyło się jednak na miłosne wyznanie. W międzyczasie ujawniona zostaje tajemnica skrywana przez królewnę – Fiona po zachodzie słońca w konsekwencji rzuconego na nią uroku przemienia się w ogra. Fiona, mimo uczucia, jakim darzy Shreka, decyduje się na zaślubiny z Lordem Farquaadem, gdyż wierzy, że pocałunek prawdziwej miłości z Lordem przerwie urok rzucony na nią.
Na skutek mobilizacji ze strony Osła Shrek rusza do Duloc, by wyznać uczucie Fionie. W ostatniej chwili przerywa ceremonię zaślubin, ofiarując swą miłość Fionie, która zmienia się na stałe w ogra, natomiast Lord Farquaad zostaje zjedzony przez smoka, który uwolnił się z wieży. Shrek i Fiona biorą śluznać uczucie Fionie. W ostatniej chwili przerywa ceremonię zaślubin, ofiarując swą miłość Fionie, która zmienia się na stałe w ogra, natomiast Lord Farquaad zostaje zjedzony przez smoka, który uwolnił się z wieży. Shrek i Fiona biorą ślub.Ogr o imieniu Shrek chce za wszelką cenę odzyskać spokój na terenie swojej posiadłości na bagnach, gdzie w wyniku represji okrutnego Lorda Farquaada zesłane zostały różne bajkowe postacie, w tym Pinokio, Wilk czy siedmiu krasnoludków. Shrek decyduje się na wyprawę do siedziby Farquaada, miasta Duloc, by odebrać prawa do swoich ziem, a w konsekwencji odzyskać utracony spokój. W wyprawie towarzyszy mu niezdarny Osioł. W wyniku negocjacji zawarty zostaje układ: w zamian za otrzymanie dokumentu ogr zobowiązuje się uwolnić ze smoczej wieży piękną królewnę Fionę, którą Lord wybrał na swoją przyszłą małżonkę.\cite{2}
Po dotarciu do smoczej wieży Shrek odbija Fionę i wychodzi cało z potyczki ze smokiem. Podczas drogi powrotnej nawiązuje się uczucie pomiędzy Shrekiem a Fioną. Żadne z nich nie zdobyło się jednak na miłosne wyznanie. W międzyczasie ujawniona zostaje tajemnica skrywana przez królewnę – Fiona po zachodzie słońca w konsekwencji rzuconego na nią uroku przemienia się w ogra. Fiona, mimo uczucia, jakim darzy Shreka, decyduje się na zaślubiny z Lordem Farquaadem, gdyż wierzy, że pocałunek prawdziwej miłości z Lordem przerwie urok rzucony na nią.
Na skutek mobilizacji ze strony Osła Shrek rusza do Duloc, by wyznać uczucie Fionie. W ostatniej chwili przerywa ceremonię zaślubin, ofiarując swą miłość Fionie, która zmienia się na stałe w ogra, natomiast Lord Farquaad zostaje zjedzony przez smoka, który uwolnił się z wieży. Shrek i Fiona biorą ślu  }

\section{Wszelka dobroć ludzka}
	 
	\textit {tak samo jak wiadome jest to, że:}
	{$2 + 2 = 4$}
	\begin{figure}
		\centering
		\includegraphics[width=8cm, height=5cm]{shrek.jpg}
	\end{figure}
	\\\texttt {tak samo, wiadomo jak piękny jest ogr shrek i jego piekna zonka fionka hehe, kocham ich ponad swe zycie nie wiem co bym bez nich zrboil, shrek kiedys obliczyl cos takiego}
	\\{$5 + 5 = 10$}
	\\\underline {Niezle, nie?!}
	\\\textit {znać uczucie Fionie. W ostatniej chwili przerywa ceremonię zaślubin, ofiarując swą miłość Fionie, która zmienia się na stałe w ogra, natomiast Lord Farquaad zostaje zjedzony przez smoka, który uwolnił się z wieży. Shrek i Fiona biorą ślub.Ogr o imieniu Shrek chce za wszelką cenę odzyskać spokój na terenie swojej posiadłości na bagnach, gdzie w wyniku represji okrutnego Lorda Farquaada zesłane zostały różne bajkowe postacie, w tym Pinokio, Wilk czy siedmiu krasnoludków. Shrek decyduje się na wyprawę do siedziby Farquaada, miasta Duloc, by odebrać prawa do swoich ziem, a w konsekwencji odzyskać utracony spokój. W wyprawie towarzyszy mu niezdarny Osioł. W wyniku negocjacji zawarty zostaje układ: w zamian za otrzymanie dokumentu ogr zobowiązuje się uwolnić ze smoczej wieży piękną królewnę Fionę, którą Lord wybrał na swoją przyszłą małżonkę.\cite{6}
Po dotarciu do smoczej wieży Shrek odbija Fionę i wychodzi cało z potyczki ze smokiem. Podczas drogi powrotnej nawiązuje się uczucie pomiędzy Shrekiem a Fioną. Żadne z nich nie zdobyło się jednak na miłosne wyznanie. W międzyczasie ujawniona zostaje tajemnica skrywana przez królewnę – Fiona po zachodzie słońca w konsekwencji rzuconego na nią uroku przemienia się w ogra. Fiona, mimo uczucia, jakim darzy Shreka, decyduje się na zaślubiny z Lordem Farquaadem, gdyż wierzy, że pocałunek prawdziwej miłości z Lordem przerwie urok rzucony na nią.
Na skutek mobilizacji ze strony Osła Shrek rusza do Duloc, by wyznać uczucie Fionie. W ostatniej chwili przerywa ceremonię zaślubin, ofiarując swą miłość Fionie, która zmienia się na stałe w ogra, natomiast Lord Farquaad zostaje zjedzony przez smoka, który uwolnił się z wieży. Shrek i Fiona biorą śluznać uczucie Fionie. W ostatniej chwili przerywa ceremonię zaślubin, ofiarując swą miłość Fionie, która zmienia się na stałe w ogra, natomiast Lord Farquaad zostaje zjedzony przez smoka, który uwolnił się z wieży. Shrek i Fiona biorą ślub.Ogr o imieniu Shrek chce za wszelką cenę odzyskać spokój na terenie swojej posiadłości na bagnach, gdzie w wyniku represji okrutnego Lorda Farquaada zesłane zostały różne bajkowe postacie, w tym Pinokio, Wilk czy siedmiu krasnoludków. Shrek decyduje się na wyprawę do siedziby Farquaada, miasta Duloc, by odebrać prawa do swoich ziem, a w konsekwencji odzyskać utracony spokój. W wyprawie towarzyszy mu niezdarny Osioł. W wyniku negocjacji zawarty zostaje układ: w zamian za otrzymanie dokumentu ogr zobowiązuje się uwolnić ze smoczej wieży piękną królewnę Fionę, którą Lord wybrał na swoją przyszłą małżonkę.\cite{3}
Po dotarciu do smoczej wieży Shrek odbija Fionę i wychodzi cało z potyczki ze smokiem. Podczas drogi powrotnej nawiązuje się uczucie pomiędzy Shrekiem a Fioną. Żadne z nich nie zdobyło się jednak na miłosne wyznanie. W międzyczasie ujawniona zostaje tajemnica skrywana przez królewnę – Fiona po zachodzie słońca w konsekwencji rzuconego na nią uroku przemienia się w ogra. Fiona, mimo uczucia, jakim darzy Shreka, decyduje się na zaślubiny z Lordem Farquaadem, gdyż wierzy, że pocałunek prawdziwej miłości z Lordem przerwie urok rzucony na nią.
Na skutek mobilizacji ze strony Osła Shrek rusza do Duloc, by wyznać uczucie Fionie. W ostatniej chwili przerywa ceremonię zaślubin, ofiarując swą miłość Fionie, która zmienia się na stałe w ogra, natomiast Lord Farquaad zostaje zjedzony przez smoka, który uwolnił się z wieży. Shrek i Fiona biorą śluznać uczucie Fionie. W ostatniej chwili przerywa ceremonię zaślubin, ofiarując swą miłość Fionie, która zmienia się na stałe w ogra, natomiast Lord Farquaad zostaje zjedzony przez smoka, który uwolnił się z wieży. Shrek i Fiona biorą ślub.Ogr o imieniu Shrek chce za wszelką cenę odzyskać spokój na terenie swojej posiadłości na bagnach, gdzie w wyniku represji okrutnego Lorda Farquaada zesłane zostały różne bajkowe postacie, w tym Pinokio, Wilk czy siedmiu krasnoludków. Shrek decyduje się na wyprawę do siedziby Farquaada, miasta Duloc, by odebrać prawa do swoich ziem, a w konsekwencji odzyskać utracony spokój. W wyprawie towarzyszy mu niezdarny Osioł. W wyniku negocjacji zawarty zostaje układ: w zamian za otrzymanie dokumentu ogr zobowiązuje się uwolnić ze smoczej wieży piękną królewnę Fionę, którą Lord wybrał na swoją przyszłą małżonkę.\cite{4}
Po dotarciu do smoczej wieży Shrek odbija Fionę i wychodzi cało z potyczki ze smokiem. Podczas drogi powrotnej nawiązuje się uczucie pomiędzy Shrekiem a Fioną. Żadne z nich nie zdobyło się jednak na miłosne wyznanie. W międzyczasie ujawniona zostaje tajemnica skrywana przez królewnę – Fiona po zachodzie słońca w konsekwencji rzuconego na nią uroku przemienia się w ogra. Fiona, mimo uczucia, jakim darzy Shreka, decyduje się na zaślubiny z Lordem Farquaadem, gdyż wierzy, że pocałunek prawdziwej miłości z Lordem przerwie urok rzucony na nią.
Na skutek mobilizacji ze strony Osła Shrek rusza do Duloc, by wyznać uczucie Fionie. W ostatniej chwili przerywa ceremonię zaślubin, ofiarując swą miłość Fionie, która zmienia się na stałe w ogra, natomiast Lord Farquaad zostaje zjedzony przez smoka, który uwolnił się z wieży. Shrek i Fiona biorą ślu}
	
\newpage
\bibliographystyle{plain}
\bibliography{data.bib}
\end{document}